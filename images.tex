\batchmode
\documentclass{beamer}
\RequirePackage{ifthen}




\mode<presentation> {


\usetheme{AnnArbor}


}


\usepackage{graphicx} 
\usepackage{caption}
\usepackage{booktabs} 
\usepackage{tikz}
\usepackage{amsmath}
\usepackage{pgfplots}
\usepackage{hyperref}


\usetikzlibrary{arrows.meta, calc}


\title[Calculus of Several Variables]{Geometric Interpretation of Derivatives in Higher Dimensions } % The short title appears at the bottom of every slide, the full title is only on the title page


\author{Hrisav Sarkar, BSC Mathematics} % Your name
\institute[Amity] % Your institution as it will appear on the bottom of every slide, may be shorthand to save space
{
Department of Mathematics, Amity University, Kolkata, India \\% Your institution for the title page
\medskip
\textit{hrisav.sarkar@s.amity.edu } % Your email address
}
\date{6 February,2025} % Date, can be changed to a custom date




\usepackage{xcolor}

\usepackage[]{inputenc}



\makeatletter

\makeatletter
\count@=\the\catcode`\_ \catcode`\_=8 
\newenvironment{tex2html_wrap}{}{}%
\catcode`\<=12\catcode`\_=\count@
\newcommand{\providedcommand}[1]{\expandafter\providecommand\csname #1\endcsname}%
\newcommand{\renewedcommand}[1]{\expandafter\providecommand\csname #1\endcsname{}%
  \expandafter\renewcommand\csname #1\endcsname}%
\newcommand{\newedenvironment}[1]{\newenvironment{#1}{}{}\renewenvironment{#1}}%
\let\newedcommand\renewedcommand
\let\renewedenvironment\newedenvironment
\makeatother
\let\mathon=$
\let\mathoff=$
\ifx\AtBeginDocument\undefined \newcommand{\AtBeginDocument}[1]{}\fi
\newbox\sizebox
\setlength{\hoffset}{0pt}\setlength{\voffset}{0pt}
\addtolength{\textheight}{\footskip}\setlength{\footskip}{0pt}
\addtolength{\textheight}{\topmargin}\setlength{\topmargin}{0pt}
\addtolength{\textheight}{\headheight}\setlength{\headheight}{0pt}
\addtolength{\textheight}{\headsep}\setlength{\headsep}{0pt}
\setlength{\textwidth}{349pt}
\newwrite\lthtmlwrite
\makeatletter
\let\realnormalsize=\normalsize
\global\topskip=2sp
\def\preveqno{}\let\real@float=\@float \let\realend@float=\end@float
\def\@float{\let\@savefreelist\@freelist\real@float}
\def\liih@math{\ifmmode$\else\bad@math\fi}
\def\end@float{\realend@float\global\let\@freelist\@savefreelist}
\let\real@dbflt=\@dbflt \let\end@dblfloat=\end@float
\let\@largefloatcheck=\relax
\let\if@boxedmulticols=\iftrue
\def\@dbflt{\let\@savefreelist\@freelist\real@dbflt}
\def\adjustnormalsize{\def\normalsize{\mathsurround=0pt \realnormalsize
 \parindent=0pt\abovedisplayskip=0pt\belowdisplayskip=0pt}%
 \def\phantompar{\csname par\endcsname}\normalsize}%
\def\lthtmltypeout#1{{\let\protect\string \immediate\write\lthtmlwrite{#1}}}%
\usepackage[tightpage,active]{preview}
\newbox\lthtmlPageBox
\newdimen\lthtmlCropMarkHeight
\newdimen\lthtmlCropMarkDepth
\long\def\lthtmlTightVBox#1#2{%
    \setbox\lthtmlPageBox\vbox{\hbox{\catcode`\_=8 #2}}%
    \lthtmlCropMarkHeight=\ht\lthtmlPageBox \advance \lthtmlCropMarkHeight 6pt
    \lthtmlCropMarkDepth=\dp\lthtmlPageBox
    \lthtmltypeout{^^J:#1:lthtmlCropMarkHeight:=\the\lthtmlCropMarkHeight}%
    \lthtmltypeout{^^J:#1:lthtmlCropMarkDepth:=\the\lthtmlCropMarkDepth:1ex:=\the \dimexpr 1ex}%
    \begin{preview}\copy\lthtmlPageBox\end{preview}}%
\long\def\lthtmlTightFBox#1#2{%
    \adjustnormalsize\setbox\lthtmlPageBox=\vbox\bgroup %
    \let\ifinner=\iffalse \let\)\liih@math %
    {\catcode`\_=8 #2}%
    \@next\next\@currlist{}{\def\next{\voidb@x}}%
    \expandafter\box\next\egroup %
    \lthtmlCropMarkHeight=\ht\lthtmlPageBox \advance \lthtmlCropMarkHeight 6pt
    \lthtmlCropMarkDepth=\dp\lthtmlPageBox
    \lthtmltypeout{^^J:#1:lthtmlCropMarkHeight:=\the\lthtmlCropMarkHeight}%
    \lthtmltypeout{^^J:#1:lthtmlCropMarkDepth:=\the\lthtmlCropMarkDepth:1ex:=\the \dimexpr 1ex}%
    \begin{preview}\copy\lthtmlPageBox\end{preview}}%
    \long\def\lthtmlinlinemathA#1#2\lthtmlindisplaymathZ{\lthtmlTightVBox{#1}{#2}}
    \def\lthtmlinlineA#1#2\lthtmlinlineZ{\lthtmlTightVBox{#1}{#2}}
    \long\def\lthtmldisplayA#1#2\lthtmldisplayZ{\lthtmlTightVBox{#1}{#2}}
    \long\def\lthtmlinlinemathA#1#2\lthtmlindisplaymathZ{\lthtmlTightVBox{#1}{#2}}
    \def\lthtmlinlineA#1#2\lthtmlinlineZ{\lthtmlTightVBox{#1}{#2}}
    \long\def\lthtmldisplayA#1#2\lthtmldisplayZ{\lthtmlTightVBox{#1}{#2}}
    \long\def\lthtmldisplayB#1#2\lthtmldisplayZ{\\edef\preveqno{(\theequation)}%
        \lthtmlTightVBox{#1}{\let\@eqnnum\relax#2}}
    \long\def\lthtmlfigureA#1#2\lthtmlfigureZ{\let\@savefreelist\@freelist
        \lthtmlTightFBox{#1}{#2}\global\let\@freelist\@savefreelist}
    \long\def\lthtmlpictureA#1#2\lthtmlpictureZ{\let\@savefreelist\@freelist
        \lthtmlTightVBox{#1}{#2}\global\let\@freelist\@savefreelist}
\def\lthtmlcheckvsize{\ifdim\ht\sizebox<\vsize 
  \ifdim\wd\sizebox<\hsize\expandafter\hfill\fi \expandafter\vfill
  \else\expandafter\vss\fi}%
\providecommand{\selectlanguage}[1]{}%
\makeatletter \tracingstats = 1 
\providecommand{\Alpha}{\textrm{A}}
\providecommand{\Eta}{\textrm{H}}
\providecommand{\Kappa}{\textrm{K}}
\providecommand{\Iota}{\textrm{J}}
\providecommand{\Mu}{\textrm{M}}
\providecommand{\Epsilon}{\textrm{E}}
\providecommand{\Beta}{\textrm{B}}
\providecommand{\Omicron}{\textrm{O}}
\providecommand{\Tau}{\textrm{T}}
\providecommand{\Rho}{\textrm{R}}
\providecommand{\Nu}{\textrm{N}}
\providecommand{\Chi}{\textrm{X}}
\providecommand{\omicron}{\textrm{o}}
\providecommand{\Zeta}{\textrm{Z}}


\begin{document}
\pagestyle{empty}\thispagestyle{empty}\lthtmltypeout{}%
\lthtmltypeout{latex2htmlLength hsize=\the\hsize}\lthtmltypeout{}%
\lthtmltypeout{latex2htmlLength vsize=\the\vsize}\lthtmltypeout{}%
\lthtmltypeout{latex2htmlLength hoffset=\the\hoffset}\lthtmltypeout{}%
\lthtmltypeout{latex2htmlLength voffset=\the\voffset}\lthtmltypeout{}%
\lthtmltypeout{latex2htmlLength topmargin=\the\topmargin}\lthtmltypeout{}%
\lthtmltypeout{latex2htmlLength topskip=\the\topskip}\lthtmltypeout{}%
\lthtmltypeout{latex2htmlLength headheight=\the\headheight}\lthtmltypeout{}%
\lthtmltypeout{latex2htmlLength headsep=\the\headsep}\lthtmltypeout{}%
\lthtmltypeout{latex2htmlLength parskip=\the\parskip}\lthtmltypeout{}%
\lthtmltypeout{latex2htmlLength oddsidemargin=\the\oddsidemargin}\lthtmltypeout{}%
\makeatletter
\if@twoside\lthtmltypeout{latex2htmlLength evensidemargin=\the\evensidemargin}%
\else\lthtmltypeout{latex2htmlLength evensidemargin=\the\oddsidemargin}\fi%
\lthtmltypeout{}%
\makeatother
\setcounter{page}{1}
\onecolumn

% !!! IMAGES START HERE !!!

{\newpage\clearpage
\lthtmlfigureA{frame16}%
\begin{frame}
\titlepage % Print the title page as the first slide
\end{frame}%
\lthtmlfigureZ
\lthtmlcheckvsize\clearpage}

\stepcounter{section}
\stepcounter{subsection}
{\newpage\clearpage
\lthtmlfigureA{frame23}%
\begin{frame}
\frametitle{Introduction}
\begin{itemize}
    \item \textbf{Overview:}
      \begin{itemize}
        \item Extends the concept of derivatives from one-dimensional calculus to functions of several variables.
        \item Provides a framework for understanding how functions change along multiple directions.
      \end{itemize}
    \item \textbf{Key Ideas:}
      \begin{itemize}
        \item \textbf{Linear Approximation:} The derivative at a point is the best linear approximation of the function near that point.
      \end{itemize}
    \item \textbf{Why It Matters:}
      \begin{itemize}
        \item \textbf{Visualization:} Offers intuitive geometric insights into the behavior of multivariable functions.
        \item \textbf{Applications:} Fundamental in optimization, solving differential equations and modeling real-world phenomena.
      \end{itemize}
  \end{itemize}
\par
\end{frame}%
\lthtmlfigureZ
\lthtmlcheckvsize\clearpage}

\stepcounter{section}
\stepcounter{subsection}
{\newpage\clearpage
\lthtmlfigureA{frame42}%
\begin{frame}
% latex2html id marker 42
[allowframebreaks]
\frametitle{Geometric Meaning of the derivative of a vector Function}
\par
The geometric meaning of the derivative of a vector function is closely related to the concept of the rate of change and the tangent vector.
\begin{itemize}
\item \textcolor{blue}{\textit{Rate of change of position}}:
\par
\begin{block}{}
    Let $r:\mathbb{R}\rightarrow \mathbb{R}^n$, where $r$\  is differentiable meaning $r(t)=(r_1(t),r_2(2),\cdots ,r_n(t))$\  and all $r_i(t)$\  such that $r_i:\mathbb{R}\rightarrow \mathbb{R}$\  are differentiable functions, and $r'(t)=(r_1'(t),r_2'(2),\cdots ,r_n'(t))$\  describes how the position of the vector-valued function $r(t)$\  changes with respect to the parameter $t$.
\end{block}
\par
\item \textcolor{blue}{\textit{Tangent Vector}}: 
\par
\begin{block}{}
    $r'(t)$\  at a specific point on the curve indicates the direction of the tangent line to the curve at that point.
\end{block}
\par
\item \textcolor{blue}{\textit{Visualisation in 3D and 2D Spaces}}: \\
\begin{block}{}
    If a particle moving along a curve, $r'(t)$\  at any point shows how the particle would continue moving if it maintains its current velocity and direction
\end{block}
\par
\end{itemize} 
Consider the unit circle defined by 
  \begin{displaymath}
  r(t) = (\cos t, \sin t).
  \end{displaymath}
  Its derivative is 
  \begin{displaymath}
  r'(t) = (-\sin t, \cos t),
  \end{displaymath}
  Which represents the tangent vector at each point.
\par
At \(t=\frac{\pi}{4}\):
  \begin{displaymath}
  r\Bigl(\frac{\pi}{4}\Bigr)=\Bigl(\frac{\sqrt{2}}{2}, \frac{\sqrt{2}}{2}\Bigr),
  \end{displaymath}
  \begin{displaymath}
  r'\Bigl(\frac{\pi}{4}\Bigr)=\Bigl(-\frac{\sqrt{2}}{2}, \frac{\sqrt{2}}{2}\Bigr).
  \end{displaymath}
\par
The tangent line is then given by
  \begin{displaymath}
  L(s)=r\Bigl(\frac{\pi}{4}\Bigr) + s\,r'\Bigl(\frac{\pi}{4}\Bigr)
  = \Bigl(\frac{\sqrt{2}}{2}, \frac{\sqrt{2}}{2}\Bigr) + s\Bigl(-\frac{\sqrt{2}}{2}, \frac{\sqrt{2}}{2}\Bigr),
  \quad s\in\mathbb{R}.
  \end{displaymath}
\begin{center}
\par
\begin{tikzpicture}[scale=2]
  % Draw coordinate axes
  \draw[->] (-1.2,0) -- (1.2,0) node[right] {$x$};
  \draw[->] (0,-1.2) -- (0,1.2) node[above] {$y$};
\par
\draw[thick,blue] (0,0) circle (1);
\par
\coordinate (P) at ({cos(45)}, {sin(45)});
\par
\filldraw[red] (P) circle (0.02);
  \node[above right] at (P) {$r(t)$};
\par
\pgfmathsetmacro{\dx}{-sin(45)}
  \pgfmathsetmacro{\dy}{cos(45)}
\par
\draw[->,red,thick] (P) -- ++({\dx/2},{\dy/2}) node[right] {$r'(t)$};
\par
\draw[dashed,red] plot[domain=-1:1] 
    ({cos(45) + \dx*\x},{sin(45) + \dy*\x});
\end{tikzpicture}
\end{center}
\captionof{figure}{A unit circle with its tangent vector and the tangent line at $t=\pi/4$.}
\end{frame}%
\lthtmlfigureZ
\lthtmlcheckvsize\clearpage}

\stepcounter{section}
\stepcounter{subsection}
{\newpage\clearpage
\lthtmlfigureA{frame116}%
\begin{frame}{Understanding the Gradient of a Function}
\par
\textbf{What is the Gradient?} \\
    The \textbf{gradient} of a function is a vector that tells us two things:
    \begin{itemize}
        \item \textbf{Direction} – Where the function increases the fastest.
        \item \textbf{Steepness} – How fast the function increases in that direction.
    \end{itemize}
\par
Mathematically, for a function \( f(x, y) \), the gradient is written as:
    \begin{displaymath}
    \nabla f = \left( \frac{\partial f}{\partial x}, \frac{\partial f}{\partial y} \right)
    \end{displaymath}
    This means:
    \begin{itemize}
        \item \( \frac{\partial f}{\partial x} \) tells how much \( f \) changes as we move in the \( x \)-direction.
        \item \( \frac{\partial f}{\partial y} \) tells how much \( f \) changes as we move in the \( y \)-direction.
    \end{itemize}
\par
\end{frame}%
\lthtmlfigureZ
\lthtmlcheckvsize\clearpage}

\stepcounter{subsection}
{\newpage\clearpage
\lthtmlfigureA{frame136}%
\begin{frame}{Example: Climbing a Hill}
    Imagine you are hiking on a hill where the height at any point is given by \( f(x, y) \).
    \begin{itemize}
        \item The \textbf{gradient} tells you the steepest path to climb up.
        \item The bigger the gradient’s length, the steeper the climb.
    \end{itemize}
\par
\textbf{Mathematical Example:} If \( f(x, y) = x^2 + y^2 \), then:
    \begin{displaymath}
    \nabla f = (2x, 2y)
    \end{displaymath}
    \begin{itemize}
        \item At \( (1,1) \), the gradient is \( (2,2) \), meaning the function increases fastest in the \((1,1)\) direction.
        \item At \( (0,0) \), the gradient is \( (0,0) \), meaning it’s a flat point (no steepest direction).
    \end{itemize}
\par
\centering
    % \includegraphics[width=0.5\linewidth]{gradienthill.png} % Add a relevant figure here
\par
\end{frame}%
\lthtmlfigureZ
\lthtmlcheckvsize\clearpage}

\stepcounter{subsection}
{\newpage\clearpage
\lthtmlfigureA{frame146}%
\begin{frame}{Geometric Meaning}
    \begin{itemize}
        \item The \textbf{gradient vector} always points in the direction where the function increases the fastest.
        \item The \textbf{steepness} (magnitude of the gradient) tells how quickly the function increases.
        \item The gradient is \textbf{perpendicular} to the level curves (contour lines).
    \end{itemize}
\end{frame}%
\lthtmlfigureZ
\lthtmlcheckvsize\clearpage}

\stepcounter{subsection}
{\newpage\clearpage
\lthtmlfigureA{frame155}%
\begin{frame}{Physical Meaning (Real-World Examples)}
    The gradient is useful in many areas of science:
    \begin{itemize}
        \item \textbf{Heat Flow:} If \( f(x, y) \) represents temperature, then \( \nabla f \) shows the direction where heat increases fastest.
        \item \textbf{Electric Fields:} The gradient of voltage tells us how electric potential changes in space.
        \item \textbf{Water Flow:} Water flows \textbf{downhill}, opposite to the gradient of elevation.
    \end{itemize}
\par
\centering
    % \includegraphics[width=0.5\linewidth]{gradient_real_world.png} % Add a relevant figure here
\par
\end{frame}%
\lthtmlfigureZ
\lthtmlcheckvsize\clearpage}

{\newpage\clearpage
\lthtmlfigureA{frame164}%
\begin{frame}
% latex2html id marker 164
[allowframebreaks]{Gradient of a Scalar-Valued Function Figure}
\par
\begin{figure}[h]
    \centering
    \begin{tikzpicture}[scale=0.6]
        % Define the function f(x, y) = x^2 + y^2
        \begin{axis}[
            view={60}{30},
            xlabel={$x$},
            ylabel={$y$},
            zlabel={$f(x,y)$},
            domain=-2:2,
            y domain=-2:2,
            samples=20,
            grid=major,
            colormap/viridis,
            colorbar,
            colorbar style={title={$f(x,y)$}},
            title={$f(x,y) = x^2 + y^2$}
        ]
            \addplot3[
                surf,
                opacity=0.7
            ] {x^2 + y^2};
        \end{axis}
\par
% Function value at (x0, y0)
          % Partial derivative df/dx at (x0, y0)
          % Partial derivative df/dy at (x0, y0)
\par
\begin{scope}[shift={(2,1)}]
            % Axes in xy-plane
            \draw[->] (-2.5, 0) -- (2.5, 0) node[right] {$x$};
            \draw[->] (0, -2.5) -- (0, 2.5) node[above] {$y$};
\par
\fill[gray!20, opacity=0.5] 
                (1-1, 1-1) -- (1+1, 1-1) -- 
                (1+1, 1+1) -- (1-1, 1+1) -- cycle;
\par
\draw[->, blue, very thick] (1, 1) -- ++(2 * 1/3, 2 * 1/3) node[right] {$\nabla f$};
\par
\draw[->, red, thick] (1, 1) -- ++(1, 0) node[right] {$\mathbf{T_1}$};
            \draw[->, red, thick] (1, 1) -- ++(0, 1) node[above] {$\mathbf{T_2}$};
\par
\node[fill=black, circle, inner sep=1.5pt] at (1, 1) {};
            \node[below left] at (1, 1) {$(x_0, y_0)$};
        \end{scope}
\par
\end{tikzpicture}
    \caption{Illustration of the function \( f(x,y) = x^2 + y^2 \), its tangent plane at \( (1,1) \), and the gradient vector perpendicular to the plane.}
\end{figure}
\par
\end{frame}%
\lthtmlfigureZ
\lthtmlcheckvsize\clearpage}

{\newpage\clearpage
\lthtmlfigureA{frame197}%
\begin{frame}
\Huge {\centerline{The End}}
\end{frame}%
\lthtmlfigureZ
\lthtmlcheckvsize\clearpage}


\end{document}
